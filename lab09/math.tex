\documentclass[12pt]{article}
\usepackage{amsmath}
\usepackage{amssymb}
\usepackage{amsthm}
\usepackage{enumerate}
\usepackage{hyperref}
\usepackage{txfonts}
\usepackage{amsmath}
\usepackage{amssymb}
\usepackage{amscd}
\usepackage{amsmath, mathtools,amssymb}
\usepackage{amsfonts,semantic,colortbl,mathrsfs,stmaryrd}
\usepackage{enumerate}
\usepackage{multirow}
\usepackage{graphicx}
\date{Feb 14, 2012}
\newtheorem{thm}{Theorem}
\newtheorem{lemma}[thm]{Lemma}
\newtheorem{fact}[thm]{Fact}
\newtheorem{cor}[thm]{Corollary}
\newtheorem{eg}{Example}
\newtheorem{hw}{Problem}
\newcommand{\xor}{\otimes}
\newenvironment{sol}
  {\par\vspace{3mm}\noindent{\it Solution}.}
  {\qed}
\begin{document}
\begin{center}
{\LARGE\bf Homework 9}\\
\vspace{2mm}
\end{center}

\begin{hw}
Explain how the volume of a ball in high dimensions can simultaneously be in a narrow slice at the equator and also be concentrated in a narrow annulus at the surface of the ball.
\end{hw}

\begin{sol}
\renewcommand{\qedsymbol}{}
We know that no matter what distribution we use, the s scattering points will near the surface because, if we build the n-dimension ball with radius $r=1$, for its volume
$$V \propto r^{n}\ ,\ \forall x<r, \lim_{n\rightarrow \infty }V'=0 $$  
Therefore, we just need to let the scattering points locate at the equator.
We know that once $x\sim N(0,1)$ random d-dimensional $x_1,x_2$ are approximately
 orthogonal. So we can let $x_{1}$ be pole, then $x_{2}$ locates at narrow slice at the equator.
\end{sol}

\begin{hw}
Find the singular value decomposition of the following matrix

$$A=\left(
  \begin{array}{ccc}
    1 & 0 & 1 \\
    0 & 1 & -1 \\
  \end{array}
\right).$$
\end{hw}

\begin{sol}
\renewcommand{\qedsymbol}{}
According to SVD, we can get
$$M=\left(
  \begin{array}{ccc}
    \frac{1}{\sqrt{2}} & \frac{1}{\sqrt{2}} \\
    \frac{-1}{\sqrt{2}} & \frac{1}{\sqrt{2}} \\
  \end{array}
\right)\times
\left(
  \begin{array}{ccc}
    \sqrt{3} & 0 & 0 \\
    0 & 1 & 0 \\
  \end{array}
\right)
\times
\left(
  \begin{array}{ccc}
    \frac{1}{\sqrt{6}} & \frac{1}{\sqrt{2}} & \frac{-1}{\sqrt{3}} \\
    \frac{-1}{\sqrt{6}} & \frac{1}{\sqrt{2}} & \frac{1}{\sqrt{3}} \\
    \frac{2}{\sqrt{6}} & 0 & \frac{1}{\sqrt{3}} \\
  \end{array}
\right)=
\left(
  \begin{array}{ccc}
    \frac{\sqrt{3}}{\sqrt{12}}-\frac{1}{\sqrt{12}} & \frac{\sqrt{3}}{2}+
    \frac{1}{2} & \frac{-1}{\sqrt{2}}+\frac{1}{\sqrt{6}} \\
    \frac{\sqrt{-3}}{\sqrt{12}}-\frac{1}{\sqrt{12}} & \frac{\sqrt{3}}{-2}+
    \frac{1}{2} & \frac{1}{\sqrt{2}}+\frac{1}{\sqrt{6}} \\
  \end{array}
\right)
$$
\end{sol}

\end{document}

%%% Local Variables:
%%% mode: tex-pdf
%%% TeX-master: t
%%% End: